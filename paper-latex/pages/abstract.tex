\chapter{\abstractname}

%TODO: Change order: first background then my work then results

Public-key cryptography has been widely employed for securing the digital world and enabling critical applications. In constrained devices, dedicated co-processors provide functionality required by popular schemes such as RSA and ECC. These schemes became obsolete with the advent of the famous Shor's algorithm that leverages the power of quantum computers to break their fundamental assumptions. Quantum-resistent schemes come as an answer to the threat of quantum computers to security by providing new cryptographic constructs that do not rely on assumptions that the Shor's algorithm can invalidate. The large scale adoption of post-quantum cryptographic schemes that were nominated to become the new standard is hindered by the fact that a large number of them require different functionality than the old dedicated cryptographic processors offer. Since these devices will still be in-use for many years to come, it is important to secure them for the future by repurposing deployed cryptographic co-processors.
 
In this thesis we implement the Kronecker+ polynomial multiplication algorithm tuned for the CRIYSTALS-Dilithium post-quantum, RLWE-based signature scheme, on the OpenTitan Big Number Accelerator (OTBN). The Kroneker+ algorithm presented by Bos et al. at USENIX 2022 builds on top of the functionality offered by common cryptographic co-processors and enables efficient polynomial multiplication which is what many new RLWE-based schemes such as Dilithium require. We benchmark the implementation for various parameter and compare the results with other implementations of the chosen scheme.